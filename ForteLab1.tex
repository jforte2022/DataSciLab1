% Options for packages loaded elsewhere
\PassOptionsToPackage{unicode}{hyperref}
\PassOptionsToPackage{hyphens}{url}
%
\documentclass[
]{article}
\usepackage{amsmath,amssymb}
\usepackage{lmodern}
\usepackage{iftex}
\ifPDFTeX
  \usepackage[T1]{fontenc}
  \usepackage[utf8]{inputenc}
  \usepackage{textcomp} % provide euro and other symbols
\else % if luatex or xetex
  \usepackage{unicode-math}
  \defaultfontfeatures{Scale=MatchLowercase}
  \defaultfontfeatures[\rmfamily]{Ligatures=TeX,Scale=1}
\fi
% Use upquote if available, for straight quotes in verbatim environments
\IfFileExists{upquote.sty}{\usepackage{upquote}}{}
\IfFileExists{microtype.sty}{% use microtype if available
  \usepackage[]{microtype}
  \UseMicrotypeSet[protrusion]{basicmath} % disable protrusion for tt fonts
}{}
\makeatletter
\@ifundefined{KOMAClassName}{% if non-KOMA class
  \IfFileExists{parskip.sty}{%
    \usepackage{parskip}
  }{% else
    \setlength{\parindent}{0pt}
    \setlength{\parskip}{6pt plus 2pt minus 1pt}}
}{% if KOMA class
  \KOMAoptions{parskip=half}}
\makeatother
\usepackage{xcolor}
\IfFileExists{xurl.sty}{\usepackage{xurl}}{} % add URL line breaks if available
\IfFileExists{bookmark.sty}{\usepackage{bookmark}}{\usepackage{hyperref}}
\hypersetup{
  pdftitle={ForteLab1.R},
  pdfauthor={andre},
  hidelinks,
  pdfcreator={LaTeX via pandoc}}
\urlstyle{same} % disable monospaced font for URLs
\usepackage[margin=1in]{geometry}
\usepackage{color}
\usepackage{fancyvrb}
\newcommand{\VerbBar}{|}
\newcommand{\VERB}{\Verb[commandchars=\\\{\}]}
\DefineVerbatimEnvironment{Highlighting}{Verbatim}{commandchars=\\\{\}}
% Add ',fontsize=\small' for more characters per line
\usepackage{framed}
\definecolor{shadecolor}{RGB}{248,248,248}
\newenvironment{Shaded}{\begin{snugshade}}{\end{snugshade}}
\newcommand{\AlertTok}[1]{\textcolor[rgb]{0.94,0.16,0.16}{#1}}
\newcommand{\AnnotationTok}[1]{\textcolor[rgb]{0.56,0.35,0.01}{\textbf{\textit{#1}}}}
\newcommand{\AttributeTok}[1]{\textcolor[rgb]{0.77,0.63,0.00}{#1}}
\newcommand{\BaseNTok}[1]{\textcolor[rgb]{0.00,0.00,0.81}{#1}}
\newcommand{\BuiltInTok}[1]{#1}
\newcommand{\CharTok}[1]{\textcolor[rgb]{0.31,0.60,0.02}{#1}}
\newcommand{\CommentTok}[1]{\textcolor[rgb]{0.56,0.35,0.01}{\textit{#1}}}
\newcommand{\CommentVarTok}[1]{\textcolor[rgb]{0.56,0.35,0.01}{\textbf{\textit{#1}}}}
\newcommand{\ConstantTok}[1]{\textcolor[rgb]{0.00,0.00,0.00}{#1}}
\newcommand{\ControlFlowTok}[1]{\textcolor[rgb]{0.13,0.29,0.53}{\textbf{#1}}}
\newcommand{\DataTypeTok}[1]{\textcolor[rgb]{0.13,0.29,0.53}{#1}}
\newcommand{\DecValTok}[1]{\textcolor[rgb]{0.00,0.00,0.81}{#1}}
\newcommand{\DocumentationTok}[1]{\textcolor[rgb]{0.56,0.35,0.01}{\textbf{\textit{#1}}}}
\newcommand{\ErrorTok}[1]{\textcolor[rgb]{0.64,0.00,0.00}{\textbf{#1}}}
\newcommand{\ExtensionTok}[1]{#1}
\newcommand{\FloatTok}[1]{\textcolor[rgb]{0.00,0.00,0.81}{#1}}
\newcommand{\FunctionTok}[1]{\textcolor[rgb]{0.00,0.00,0.00}{#1}}
\newcommand{\ImportTok}[1]{#1}
\newcommand{\InformationTok}[1]{\textcolor[rgb]{0.56,0.35,0.01}{\textbf{\textit{#1}}}}
\newcommand{\KeywordTok}[1]{\textcolor[rgb]{0.13,0.29,0.53}{\textbf{#1}}}
\newcommand{\NormalTok}[1]{#1}
\newcommand{\OperatorTok}[1]{\textcolor[rgb]{0.81,0.36,0.00}{\textbf{#1}}}
\newcommand{\OtherTok}[1]{\textcolor[rgb]{0.56,0.35,0.01}{#1}}
\newcommand{\PreprocessorTok}[1]{\textcolor[rgb]{0.56,0.35,0.01}{\textit{#1}}}
\newcommand{\RegionMarkerTok}[1]{#1}
\newcommand{\SpecialCharTok}[1]{\textcolor[rgb]{0.00,0.00,0.00}{#1}}
\newcommand{\SpecialStringTok}[1]{\textcolor[rgb]{0.31,0.60,0.02}{#1}}
\newcommand{\StringTok}[1]{\textcolor[rgb]{0.31,0.60,0.02}{#1}}
\newcommand{\VariableTok}[1]{\textcolor[rgb]{0.00,0.00,0.00}{#1}}
\newcommand{\VerbatimStringTok}[1]{\textcolor[rgb]{0.31,0.60,0.02}{#1}}
\newcommand{\WarningTok}[1]{\textcolor[rgb]{0.56,0.35,0.01}{\textbf{\textit{#1}}}}
\usepackage{graphicx}
\makeatletter
\def\maxwidth{\ifdim\Gin@nat@width>\linewidth\linewidth\else\Gin@nat@width\fi}
\def\maxheight{\ifdim\Gin@nat@height>\textheight\textheight\else\Gin@nat@height\fi}
\makeatother
% Scale images if necessary, so that they will not overflow the page
% margins by default, and it is still possible to overwrite the defaults
% using explicit options in \includegraphics[width, height, ...]{}
\setkeys{Gin}{width=\maxwidth,height=\maxheight,keepaspectratio}
% Set default figure placement to htbp
\makeatletter
\def\fps@figure{htbp}
\makeatother
\setlength{\emergencystretch}{3em} % prevent overfull lines
\providecommand{\tightlist}{%
  \setlength{\itemsep}{0pt}\setlength{\parskip}{0pt}}
\setcounter{secnumdepth}{-\maxdimen} % remove section numbering
\ifLuaTeX
  \usepackage{selnolig}  % disable illegal ligatures
\fi

\title{ForteLab1.R}
\author{andre}
\date{2022-05-21}

\begin{document}
\maketitle

\begin{Shaded}
\begin{Highlighting}[]
\CommentTok{\# Boolean logic is used to determine whether various statements are true or false. These are the only two }
\CommentTok{\# options in boolean logic. TRUE is equivalent to 1 and FALSE is equivalent to 0 in boolean logic. }
\CommentTok{\# There are three operations that can be applied in boolean logic. These operations are AND, OR, and NOT. I}
\CommentTok{\# will demonstrate these operations in the following examples:}

\CommentTok{\# First, let\textquotesingle{}s discuss the AND operation. It is denoted by "\&" in R. For an operation to result in TRUE }
\CommentTok{\# using the AND operation, both operands (the things on both sides of the AND operation) must be TRUE. }
\CommentTok{\# Here are the four possible outcomes:}

\ConstantTok{TRUE} \SpecialCharTok{\&} \ConstantTok{TRUE}
\end{Highlighting}
\end{Shaded}

\begin{verbatim}
## [1] TRUE
\end{verbatim}

\begin{Shaded}
\begin{Highlighting}[]
\ConstantTok{TRUE} \SpecialCharTok{\&} \ConstantTok{FALSE}
\end{Highlighting}
\end{Shaded}

\begin{verbatim}
## [1] FALSE
\end{verbatim}

\begin{Shaded}
\begin{Highlighting}[]
\ConstantTok{FALSE} \SpecialCharTok{\&} \ConstantTok{TRUE}
\end{Highlighting}
\end{Shaded}

\begin{verbatim}
## [1] FALSE
\end{verbatim}

\begin{Shaded}
\begin{Highlighting}[]
\ConstantTok{FALSE} \SpecialCharTok{\&} \ConstantTok{FALSE}
\end{Highlighting}
\end{Shaded}

\begin{verbatim}
## [1] FALSE
\end{verbatim}

\begin{Shaded}
\begin{Highlighting}[]
\CommentTok{\# Now let\textquotesingle{}s discuss the OR operation. It is denoted by "|" in R. For an operation to result in TRUE using}
\CommentTok{\# the OR operation, either one of the operands or both must be TRUE. It only results in FALSE if both}
\CommentTok{\# operands are FALSE. Here are the possible outcomes:}

\ConstantTok{TRUE} \SpecialCharTok{|} \ConstantTok{TRUE}
\end{Highlighting}
\end{Shaded}

\begin{verbatim}
## [1] TRUE
\end{verbatim}

\begin{Shaded}
\begin{Highlighting}[]
\ConstantTok{TRUE} \SpecialCharTok{|} \ConstantTok{FALSE}
\end{Highlighting}
\end{Shaded}

\begin{verbatim}
## [1] TRUE
\end{verbatim}

\begin{Shaded}
\begin{Highlighting}[]
\ConstantTok{FALSE} \SpecialCharTok{|} \ConstantTok{TRUE}
\end{Highlighting}
\end{Shaded}

\begin{verbatim}
## [1] TRUE
\end{verbatim}

\begin{Shaded}
\begin{Highlighting}[]
\ConstantTok{FALSE} \SpecialCharTok{|} \ConstantTok{FALSE}
\end{Highlighting}
\end{Shaded}

\begin{verbatim}
## [1] FALSE
\end{verbatim}

\begin{Shaded}
\begin{Highlighting}[]
\CommentTok{\# The final operation in boolean logic is NOT. It is denoted by "!" in R. The NOT operation simply negates the}
\CommentTok{\# result of a statement. Here are some examples:}

\SpecialCharTok{!}\ConstantTok{TRUE}
\end{Highlighting}
\end{Shaded}

\begin{verbatim}
## [1] FALSE
\end{verbatim}

\begin{Shaded}
\begin{Highlighting}[]
\SpecialCharTok{!}\ConstantTok{FALSE}
\end{Highlighting}
\end{Shaded}

\begin{verbatim}
## [1] TRUE
\end{verbatim}

\begin{Shaded}
\begin{Highlighting}[]
\SpecialCharTok{!}\NormalTok{(}\ConstantTok{TRUE} \SpecialCharTok{\&} \ConstantTok{TRUE}\NormalTok{)}
\end{Highlighting}
\end{Shaded}

\begin{verbatim}
## [1] FALSE
\end{verbatim}

\begin{Shaded}
\begin{Highlighting}[]
\CommentTok{\# Take note of how the operations can be combined to form more complex statements. TRUE \& TRUE should}
\CommentTok{\# result in TRUE, but is negated by the NOT operation and, therefore, results in FALSE. Also remember that}
\CommentTok{\# TRUE can be replaced by a 1 and FALSE can be replaced by  a 0.}

\CommentTok{\# The following conditional statements will tell you whether the expected outcomes occur,}
\CommentTok{\# or if there is an error in the statement (I have replaced TRUE with 1 and FALSE with 0):}

\ControlFlowTok{if}\NormalTok{( }\DecValTok{1}\SpecialCharTok{\&}\DecValTok{1}\NormalTok{ ) }\StringTok{"1\&1 is true"} \ControlFlowTok{else} \StringTok{"error somewhere"}
\end{Highlighting}
\end{Shaded}

\begin{verbatim}
## [1] "1&1 is true"
\end{verbatim}

\begin{Shaded}
\begin{Highlighting}[]
\ControlFlowTok{if}\NormalTok{( }\DecValTok{1}\SpecialCharTok{\&}\DecValTok{0}\NormalTok{ ) }\StringTok{"error somewhere"} \ControlFlowTok{else} \StringTok{"1\&0 is false"}
\end{Highlighting}
\end{Shaded}

\begin{verbatim}
## [1] "1&0 is false"
\end{verbatim}

\begin{Shaded}
\begin{Highlighting}[]
\ControlFlowTok{if}\NormalTok{( }\DecValTok{0}\SpecialCharTok{\&}\DecValTok{1}\NormalTok{ ) }\StringTok{"error somewhere"} \ControlFlowTok{else} \StringTok{"0\&1 is false"}
\end{Highlighting}
\end{Shaded}

\begin{verbatim}
## [1] "0&1 is false"
\end{verbatim}

\begin{Shaded}
\begin{Highlighting}[]
\ControlFlowTok{if}\NormalTok{( }\DecValTok{0}\SpecialCharTok{\&}\DecValTok{0}\NormalTok{ ) }\StringTok{"error somewhere"} \ControlFlowTok{else} \StringTok{"0\&0 is false"}
\end{Highlighting}
\end{Shaded}

\begin{verbatim}
## [1] "0&0 is false"
\end{verbatim}

\begin{Shaded}
\begin{Highlighting}[]
\ControlFlowTok{if}\NormalTok{( }\DecValTok{1}\SpecialCharTok{|}\DecValTok{1}\NormalTok{ ) }\StringTok{"1|1 is true"} \ControlFlowTok{else} \StringTok{"error somewhere"}
\end{Highlighting}
\end{Shaded}

\begin{verbatim}
## [1] "1|1 is true"
\end{verbatim}

\begin{Shaded}
\begin{Highlighting}[]
\ControlFlowTok{if}\NormalTok{( }\DecValTok{1}\SpecialCharTok{|}\DecValTok{0}\NormalTok{ ) }\StringTok{"1|0 is true"} \ControlFlowTok{else} \StringTok{"error somewhere"}
\end{Highlighting}
\end{Shaded}

\begin{verbatim}
## [1] "1|0 is true"
\end{verbatim}

\begin{Shaded}
\begin{Highlighting}[]
\ControlFlowTok{if}\NormalTok{( }\DecValTok{0}\SpecialCharTok{|}\DecValTok{1}\NormalTok{ ) }\StringTok{"0|1 is true"} \ControlFlowTok{else} \StringTok{"error somewhere"}
\end{Highlighting}
\end{Shaded}

\begin{verbatim}
## [1] "0|1 is true"
\end{verbatim}

\begin{Shaded}
\begin{Highlighting}[]
\ControlFlowTok{if}\NormalTok{( }\DecValTok{0}\SpecialCharTok{|}\DecValTok{0}\NormalTok{ ) }\StringTok{"error somewhere"} \ControlFlowTok{else} \StringTok{"0|0 is false"}
\end{Highlighting}
\end{Shaded}

\begin{verbatim}
## [1] "0|0 is false"
\end{verbatim}

\begin{Shaded}
\begin{Highlighting}[]
\ControlFlowTok{if}\NormalTok{(}\SpecialCharTok{!}\DecValTok{1}\NormalTok{) }\StringTok{"error somewhere"} \ControlFlowTok{else} \StringTok{"!1 is false"}
\end{Highlighting}
\end{Shaded}

\begin{verbatim}
## [1] "!1 is false"
\end{verbatim}

\begin{Shaded}
\begin{Highlighting}[]
\ControlFlowTok{if}\NormalTok{(}\SpecialCharTok{!}\DecValTok{0}\NormalTok{) }\StringTok{"!0 is true"} \ControlFlowTok{else} \StringTok{"error somewhere"}
\end{Highlighting}
\end{Shaded}

\begin{verbatim}
## [1] "!0 is true"
\end{verbatim}

\begin{Shaded}
\begin{Highlighting}[]
\ControlFlowTok{if}\NormalTok{(}\SpecialCharTok{!}\NormalTok{(}\ConstantTok{TRUE} \SpecialCharTok{\&} \ConstantTok{TRUE}\NormalTok{)) }\StringTok{"error somewhere"} \ControlFlowTok{else} \StringTok{"!(TRUE \& TRUE) is false"}
\end{Highlighting}
\end{Shaded}

\begin{verbatim}
## [1] "!(TRUE & TRUE) is false"
\end{verbatim}

\begin{Shaded}
\begin{Highlighting}[]
\CommentTok{\# The following code demonstrates various methods of vector math}

\NormalTok{height }\OtherTok{\textless{}{-}} \FunctionTok{c}\NormalTok{(}\DecValTok{59}\NormalTok{,}\DecValTok{60}\NormalTok{,}\DecValTok{61}\NormalTok{,}\DecValTok{58}\NormalTok{,}\DecValTok{67}\NormalTok{,}\DecValTok{72}\NormalTok{,}\DecValTok{70}\NormalTok{)}
\NormalTok{weight }\OtherTok{\textless{}{-}} \FunctionTok{c}\NormalTok{(}\DecValTok{150}\NormalTok{,}\DecValTok{140}\NormalTok{,}\DecValTok{180}\NormalTok{,}\DecValTok{220}\NormalTok{,}\DecValTok{160}\NormalTok{,}\DecValTok{140}\NormalTok{,}\DecValTok{130}\NormalTok{)}
\NormalTok{a }\OtherTok{\textless{}{-}} \DecValTok{150}

\FunctionTok{mean}\NormalTok{(height)}
\end{Highlighting}
\end{Shaded}

\begin{verbatim}
## [1] 63.85714
\end{verbatim}

\begin{Shaded}
\begin{Highlighting}[]
\FunctionTok{mean}\NormalTok{(weight)}
\end{Highlighting}
\end{Shaded}

\begin{verbatim}
## [1] 160
\end{verbatim}

\begin{Shaded}
\begin{Highlighting}[]
\FunctionTok{length}\NormalTok{(}\FunctionTok{c}\NormalTok{(height, weight))}
\end{Highlighting}
\end{Shaded}

\begin{verbatim}
## [1] 14
\end{verbatim}

\begin{Shaded}
\begin{Highlighting}[]
\FunctionTok{sum}\NormalTok{(height)}
\end{Highlighting}
\end{Shaded}

\begin{verbatim}
## [1] 447
\end{verbatim}

\begin{Shaded}
\begin{Highlighting}[]
\FunctionTok{sum}\NormalTok{(height) }\SpecialCharTok{/} \FunctionTok{length}\NormalTok{(height)}
\end{Highlighting}
\end{Shaded}

\begin{verbatim}
## [1] 63.85714
\end{verbatim}

\begin{Shaded}
\begin{Highlighting}[]
\NormalTok{maxH }\OtherTok{\textless{}{-}} \FunctionTok{max}\NormalTok{(height)}
\NormalTok{minW }\OtherTok{\textless{}{-}} \FunctionTok{min}\NormalTok{(weight)}

\NormalTok{newWeight }\OtherTok{\textless{}{-}}\NormalTok{ weight }\SpecialCharTok{+} \DecValTok{5}
\NormalTok{newWeight }\SpecialCharTok{/}\NormalTok{ height}
\end{Highlighting}
\end{Shaded}

\begin{verbatim}
## [1] 2.627119 2.416667 3.032787 3.879310 2.462687 2.013889 1.928571
\end{verbatim}

\begin{Shaded}
\begin{Highlighting}[]
\ControlFlowTok{if}\NormalTok{(maxH }\SpecialCharTok{\textgreater{}} \DecValTok{60}\NormalTok{) }\StringTok{"yes"} \ControlFlowTok{else} \StringTok{"no"}
\end{Highlighting}
\end{Shaded}

\begin{verbatim}
## [1] "yes"
\end{verbatim}

\begin{Shaded}
\begin{Highlighting}[]
\ControlFlowTok{if}\NormalTok{(minW }\SpecialCharTok{\textgreater{}}\NormalTok{ a) }\StringTok{"yes"} \ControlFlowTok{else} \StringTok{"no"}
\end{Highlighting}
\end{Shaded}

\begin{verbatim}
## [1] "no"
\end{verbatim}

\end{document}
